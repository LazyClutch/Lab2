\section*{Task 2: Malware}
\begin{enumerate}
\item % 2 (a)
  \begin{itemize}
  \item A backdoor is a mechanism that allows a user to circumvent the authentication process of a system, a typical example of which is the presence of a predefined username that requires no password during authentication. While backdoors could be handy for legitimate administrative purposes, it should be clear that they are devastating in the hands of an attacker. It should be noted that backdoors need not be code snippets planted in an authentication process but rather could be standalone programs. Such a program, commonly referred to as a RAT (Remote Access Trojan) when used malevolently, could be installed by malware without the user's consent and give an attacker access to the system.\cite{aycock06}
  \end{itemize}
  Backdoor: Something that allows unauthorized users access to a system\\
  Bot: Remotely controlled computer. Several bots form a botnet\\
  Keylogger: Logs and sends (to attacker) key presses (and possibly related things like clipboard information). Keyloggers are an example of spyware\\
  Spyware: Software that monitors system\\
  Rootkit: Malware that modifies the victim's system in order to hide from detection\\

  Rootkit combined with anything makes the malware hard to detect
\item % 2 (b)
  Suppose an attacker is combining a keylogger, spyware, and rootkit
Intrusion detection system that monitors the network traffic can detect an attack in despite the rootkit
Once the attack is detected, the malware will still be hard to remove
One option is to boot a clean operating system (from a disc or USB stick) and have it removed from there
  
\end{enumerate}
