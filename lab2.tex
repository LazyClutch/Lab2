\documentclass{article}
\bibliographystyle{ieeetran}

% The first of these two packages allow for accented characters to be copied
% from the resulting PDF properly, and the second package allows for accented
% characters to be entered in the TeX file
\usepackage[T1]{fontenc}
\usepackage[utf8]{inputenc}

% Clickable hyperlinks in the report with \url
\usepackage[hyphens]{url}

% Makes it easy to refer to listings, figures, and the likes of them with
% \autoref{labelName}
\usepackage{hyperref}

% Enable to insert images
\usepackage{graphicx} 

% The listings package is useful for including source code in the report.
% It provides (among other things):
%   * lstlisting environment for listing code directly
%   * lstinputlistings command for listing code from a file
%   * lstinline command for code appearing in the middle of a sentence
\usepackage{listings}
\usepackage{xcolor}

% Increases the chance that things will fit on a line properly :)
\usepackage{fullpage}

% Make enumerate go (a), (b), ... instead of 1., 2., ...
\renewcommand{\labelenumi}{(\alph{enumi})}

% ... and then I., II., ... instead of (a), (b), ...
\renewcommand{\labelenumii}{\Roman{enumii}.}

% The enumerate package and \begin{enumerate}[(a)] or the likes of it can be
% used to break the norm though
\usepackage{enumerate}

% Avoids "code" duplication
\newcommand{\highergradesonly}{[\textbf{higher grades only}]}

% Possibly useful things for making the text more readable
\newcommand{\program}[1]{#1}
\newcommand{\exploitname}[1]{\textit{#1}}
\newcommand{\msexploit}[1]{#1}
\newcommand{\option}[1]{#1}
\newcommand{\payloadname}[1]{#1}
\newcommand{\ipaddress}[1]{#1}
\newcommand{\filename}[1]{#1}

\title{\textbf{Secure Computer Systems I: Lab 2}}
\author{Ren Li \and Tianyao Ma \and Samuel Pettersson}

\begin{document}
\maketitle
\section*{Task 1: Networking}
\begin{enumerate}
\item % 1 (a)
  The fundamental security properties that IPsec offers are integrity and confidentiality.
   \par Intergrity guarantees that data will not be changed during transmission, which means that the receiver will receive exactly the same thing sent from sender. Authentication Headers (AH) and Encapsulating Security Payloads (ESP) are two protocols that provide integrity.\cite{wikipedia_IPSec}
   \par Confidentiality guarantees that only the receiver himself has the ability to disclose the data. To do this, data has to be encrypted before transmission, which can be done by using ESP. To decrypt, the receiver should have a shared key with the sender.
  %TODO: Briefly explain how IPsec offers integrity and confidentiality
  \item \highergradesonly
  \item % 1 (c)
    \begin{enumerate}
    \item
      BHash, compared with Basic HTTP authentication, gains more security in that Basic HTTP authentication, username and password are encoded with base-64\cite{wikipedia_BasicHTTP}, which can be decoded by certain algorithm. Thus, it's not safe when attacker get the message. BHash, on the other hand, improve its security by hashing the username and password before base-64 encoding. Even the attacker get the message and decode it, he can't find any useful information.  
    \item
    As for BHash, it is vulnerable to replay attacks. As the attack hold the message, he can send it and pretend that he is the original sender. The receiver will trust the attacker because there is nothing else that can be done to check whether the message is sent from authentic sender.
    Also, BHash is vulnerable to dictionary attacks or attacks with lookup tables for reversing the hash which means that there is possibility for the attacker to acquire the username and password.
   \par Digest HTTP authentication, on the contrary, gains security in a more sophisticated and security way. Username, password and realm (a description of the computer or system being accessed) will firstly be hashed. The hashed data will then be hashed together with nouce and cnonce. Nonce is a single-use value every session with a timestamp to prevent replay attacks and chosen-plaintext attack.\cite{wikipedia_DigestHTTP} Thus, it's almost impossible for attacker to forge the message or decode it.
    \end{enumerate}
\end{enumerate}

\section*{Task 2: Malware}
\begin{enumerate}
\item % 2 (a)
  \begin{itemize}
  \item A backdoor is a mechanism that allows a user to circumvent the authentication process of a system, a typical example of which is the existence of a predefined username that requires no password during authentication. While backdoors could be handy for legitimate administrative purposes, it should be clear that they are devastating in the hands of an attacker. It should be noted that backdoors need not be code snippets planted in an authentication process but rather could be standalone programs. Such a program, commonly referred to as a RAT (Remote Access Trojan) when used malevolently, could be installed by malware without the user's consent and give an attacker access to the system.\cite{aycock06}
  \item
    A bot is an application that allows for the automation of tasks usually carried out by humans. Legitimate examples of bots are web crawlers that collect information about the web and IRC (Internet Relay Chat) bots that manage chat servers\cite{cisco_difference}. More iniquitous bots, falling into the category of malware, may be designed to infect other hosts and report back to a command and control server. Through this server, a botmaster may instruct the network of infected hosts---the \textit{botnet}---to carry out horrendous deeds such as (distributed) denial of service attacks and sending spam\cite{cisco_botnets}.
  \item
    A keylogger is used for logging the keystrokes on a keyboard. They can be implemented in either software or hardware.

Keyloggers implemented in software may be designed to operate in the kernel, through which all the keystrokes are passed, and silently log the keystrokes to a local file. The file may then periodically be sent by e-mail for analysis by the attacker\cite{wikipedia_keylogger}. It should be noted that software keyloggers often record more than just keystrokes. Mouse actions, clipboard operations, the title of the focused window, and screenshots are just some examples of information that is particularly valuable when combined with keystroke data\cite{sagiroglu09}.

A common type of hardware keyloggers is an adapter-like device placed between the I/O port on the computer and the keyboard cable, which simply forwards the keystrokes to the computer while at the same time logging them to its internal memory.

The information gathered by the keyloggers can be used in several ways of varying legitimacy: gathering secret information like passwords, identity theft, intrusion detection, and parental monitoring\cite{sagiroglu09}.

Capturing electromagnetic emissions of a wired keyboard is a type of keylogging that is too interesting not to mention. Keylogging of that type has been shown to be possible to be carried out successfully on unmodified keyboards of various models at a distance of 20 meters\cite{vuagnoux10}.
  \item
    Spyware is malware that surreptitiously collects information about a user and his or her system. The information gathered may include but is not limited to usernames and passwords (either found stored on the computer or using keyloggers as mentioned earlier), e-mail addresses for spamming purposes, and bank account and credit card numbers.\cite{aycock06}
  \item A rootkit is a collection of tools that hides its presence or the presence of other software, typically malware, from detection. This is done by modifying system software that would otherwise be able to detect that which is being hidden. It should be noted that enabling administrator access for an attacker was originally a defining characteristic of a rootkit (hence the name \emph{root}kit) but that the importance of that characteristic has diminished.\cite{mcafee06}
  \end{itemize}
  A rootkit combined with mostly anything makes for a piece of malware that is difficult not only to detect but also to remove once detected. One could imagine that a rootkit combined with a software keylogger that only occasionally sends the logged keys from the infected system would be nigh impossible to detect and could end up running uninterrupted for the lifetime of that system.
\item % 2 (b)
  If the rootkit hides the keylogger properly, there is little a system administrator can do to even detect the malware from the infected system. By running an intrusion detection system (IDS) on another machine, it might be possible to detect the occasional e-mail sent by the keylogger (assuming that that is how the keylogger transfers the recorded data) and identify it as caused by a keylogger. Removing the malware could still prove difficult or even practically impossible if the rootkit has infected the kernel. The simplest solution might be to reinstall the operating system.\cite{wikipedia_rootkit}  
\end{enumerate}

\section*{Task 3: Intrusion detection}
\begin{enumerate}
\item % 3 (a)
\item \highergradesonly
\end{enumerate}

\section*{Task 4: Buffer overflows}
In the fourth task we were required to carry out an attack against another machine in the pen test lab. [Metasploit was used. Metasploit was read about at <website>]
\begin{enumerate}
\item % 4 (a)
  There are three interfaces to the Metasploit Framework: a CLI, a GUI, and a console interface\cite{}. We used the console interface, which is started by issuing the command \lstinline{msfconsole} in a terminal window.
  The first step in developing the attack on the specific host was to scan its ports for services running. This was done using the program \program{nmap} as follows: \lstinline{nmap -sS -v -A -p1-1024 192.168.2.133}. The most important part of the output is shown below.
[Output]
The output shows that the target machine is running (among other things) a service called Samba on port 139.

In order to find a suitable vulnerability in the service, the version of Samba had to be determined. This was done by using a Samba version scanner included in the Metasploit Framework library, following the example on the website [clarify what website]. Specifically, the following three lines were entered in the console:
\begin{itemize}
\item \lstinline{use auxiliary/scanner/smb/smb_version}
\item \lstinline{set RHOSTS 192.168.2.133}
\item \lstinline{run}
\end{itemize}
The first line selects the scanner module, the second line sets the target to be scanned to the machine that we are intending to attack, and the third line runs the scanner. The output is shown below.
[Output]
The output shows that the version of the Samba service running is 2.2.1a.

The next step is to search for a vulnerability and an exploit for that version of Samba. We made use of The Exploit Database (see \url{http://www.exploit-db.com}) for this. Specifically, a free-text search of ``samba 2.2.1a'' gave one hit, namely, \exploitname{Samba trans2open Overflow (Linux x86)}. Searching for trans2open in the Metasploit console then revealed the name of an exploit module for that very exploit: \msexploit{exploit/linux/samba/trans2open}.

The following step was to carry out the attack using the exploit in Metasploit. It was done by issuing the following commands:
\begin{enumerate}[1.]
\item \lstinline{use exploit/linux/samba/trans2open}
\item \lstinline{show options}
\item \lstinline{set RHOST 192.168.2.133}
\item \lstinline{show payloads}
\item \lstinline{set payload linux/x86/shell_bind_tcp}
\item \lstinline{exploit}
\end{enumerate}
The first line selects the exploit module for the \exploitname{trans2open} exploit, the second shows the options for the module. The only required option for the exploit that was empty was \option{RHOST}, which specifies the target address. The third line sets said option to \ipaddress{192.168.2.133}---the address of the target machine. The fourth line then shows the payloads available for the exploit, and the fifth line selects a payload called \payloadname{Bind TCP Inline} [explain what the payload does]. Finally, the last line executes the attack.

Upon execution, the following is printed in the console.
[Output of the attack]
[Explanation of the output]

Having achieved root access to the system, we found a text file called \filename{secret.txt} in the home folder of the root user. Its content was a marvelous piece of ASCII art with text instructing us to include the following code in the report: C14Hzd1hkNQ2w.
\item \highergradesonly
\end{enumerate}

\section*{Task 5: Access control}
\begin{enumerate}
\item % 5 (a)
  Capabilities are one way of representing an access control matrix, which describes the protection state of a system. Each subject (user) of the system maintains a list of access rights to the objects (resources) of the system. Such a list is referred to as a capability. The concept of capabilities can be contrasted with that of access control lists, where a list of access rights for the subjects are maintained for each object. An important difference between capabilities and access control lists is that the user typically maintains his or her capability and presents it to the system during access control, whereas access control lists are handled entirely by the system. Capabilities must therefore be protected against forgery. One way of implementing such protection is through the use of cryptography.\cite{bishop02}

Implementing capabilities using web cookies and SSH keys can be done as follows. The server can encrypt the capability of a user with its private key and then send it to the user as a cookie. When the user makes a request, the cookie is sent to the server (that is, the capability is presented to the server), decrypted with the server's public key, and checked to make sure that the user is authorized to make the request. Because the private SSH key used to create the capabilities is known only to the server and it is computationally intractable for the user to determine the key, only the server can create these encrypted capabilities; malevolent users cannot forge a capability with specific rights.

Another, more scalable way of implementing the capabilities would be to have the cookies represent messages digitally signed by the server. That is, the cookies would be of the form $c + \textit{sig}\big(h(c)\big)$, where $c$ is the capability of the user, \textit{sig} is the signature algorithm used by the server with its private key, and $h$ is a cryptographic hash function. When the server receives a request with a corresponding cookie, the signature is verified with the public key of the server by comparing it with the hash of the message. If the signature is genuine, the capability is accepted and examined. If the capability specifies the right to access the requested object, access is finally granted.
\item % 5 (b)
\item \highergradesonly
\end{enumerate}

\begin{thebibliography}{9}
% Replace this with real references
\bibitem{website:wikipedia_CSRF_XSS}
  Wikipedia,
  ``Cross-site request forgery'',
  accessed 2014--02--01.
  [Online]
  Available: \url{http://en.wikipedia.org/wiki/Cross-site_request_forgery}
\end{thebibliography}
\end{document}
